\documentclass{article}

\setlength{\hoffset}{0pt}
\setlength{\textwidth}{6.5in}
\setlength{\oddsidemargin}{0pt}
\setlength{\marginparsep}{0pt}
\setlength{\marginparwidth}{0pt}

% fix section numbering to conform to KPA.## template
\renewcommand{\thesection}
{
    KPA.
    \ifnum \value{section}<10 0\fi
    \arabic{section}
}

% Squrill away Key Process Area name for use in KPARate
\newcommand{\KPAname}{}
\let\KPAsection=\section
\renewcommand{\section}[1]{\renewcommand{\KPAname}{#1}\KPAsection{#1}}

% Key Process Activity environment
\newcounter{activity}		% KPA counter
\newenvironment{KPAActivity}
{
    \setcounter{activity}{0} % reset counter
    {\bf Activity Indicators (rate your organization's capability
    on an integer scale from 0 to 10)} % header
    \begin{center}
    \begin{tabular}{|p{0.5in}|p{6.0in}|} \hline % table
}
{
    \end{tabular}
    \end{center}
}
\newcommand{\Activity}[2]
{
    \stepcounter{activity} #1 & \arabic{activity}. #2 \\ \hline
} % step counter, print score, text

% Key Process Area Score
\newenvironment{KPARate}
{
    {\bf Rate your approach, deployment, and results for
    this process area on an integer scale from 0 to 10, and 
    identify evidence below that will be provided to support your
    ratings for this process area.}
    \begin{center}
    \begin{tabular}{|p{1.0in}|p{0.5in}|p{5.0in}|} \hline
    & Rating (0-10) & \multicolumn{1}{c|}{Evidence} \\ \hline
}
{
    \end{tabular}
    \end{center}
}

\newcommand{\Approach}[2]{Approach to \KPAname & #1 & #2 \\ \hline} 
\newcommand{\Deployment}[2]{Deployment of \KPAname & #1 & #2 \\ \hline} 
\newcommand{\Results}[2]{Results Achieved in \KPAname & #1 & #2 \\
	\hline}

\newenvironment{KPAComment}
{
    {\bf Provide comments on other approaches, plans,
    or applicability of this process area to your organization's
    environment.  Use a seperate sheet if prefered.}
    \begin{center}
    \begin{tabular}{|p{6.50in}|} \hline
}
{
    \\ \hline
    \end{tabular}
    \end{center}
}

\title{Software Process Maturity Questionnaire} 
\pagestyle{myheadings}
\markright{Draco 99 Software Process Maturity Questionnaire} 

\begin{document}
\maketitle
\begin{center}
\parbox{35ex}{X--TM, MS D409\\
Los Alamos National Laboratory\\
Los Alamos, NM 87545\\
FAX 5-5538\\}

\begin{tabular}{|l|r|l|} \hline
{\bf Respondents' Names} & {\bf Work Phone} & {\bf E-mail} \\ \hline
John McGhee              & 7-9552           & mcghee@lanl.gov \\
Denise Archuleta         & 5-1855           & dga@lanl.gov \\
Mark Gray                & 7-5341           & gray@lanl.gov \\
Randy Roberts            & 5-4285           & rsqrd@lanl.gov \\
Shawn Pautz              & 7-9138           & pautz@lanl.gov \\
Tom Evans                & 5-3677           & tme@lanl.gov \\
Todd Urbatsch            & 7-3513           & tmonster@lanl.gov \\ \hline
\end{tabular}
\end{center}

\newpage
\section{Requirements Management}

\begin{KPAActivity}
\Activity{3}{The software engineering group reviews the allocated
requirements before they are incorporated into the software project.}
% We do this informally now through team meetings and hallway
% discussions, and are scheduling classes with Gerald Weinberg to do
% formal technical reviews.  Handbook of Walkthroughs, Inspections, 
% and Technical Reviews; Exploring Requirements
\Activity{3}{The software engineering group used the allocated
requirements as the basis for software plans, work products, and
activities.}
% PIP exercise, MPW course.
\Activity{3}{Changes to the allocated requirements are reviewed and
incorporated into the software project.}
% Revision after Geoff's departure.
\end{KPAActivity}

\begin{KPARate}
\Approach{3}{Meetings set up by team to establish requirements, Memos
documenting requirements, Weinburg's Exporing Requirements, Software
guidlines (Meyer, Meyers, STL, Lakos), Software quality guidelines
(LA-UR-5735), Configuration management guidelines (GNU standard, Build
System Document)}
\Deployment{3}{Memos documenting requirements}
\Results{2}{Use by management for allocations, etc.}
\end{KPARate}

\newpage
\section{Software Project Planning}

\begin{KPAActivity}
\Activity{4}{The software engineering group participates on the
project proposal team.}
% MTP exercise, MPW.
\Activity{3}{Software project planning is initiated in the early
stages of, and in parallel with, the overall project planning.}
% MTP exercise, MPW.
\Activity{3}{The software engineering group participates with other
affected groups in the overall project planning throughout the
project's life.}
% MTP exercise, MPW.
\Activity{1}{Software project commitments made to individual groups
external to the organization are reviewed with senior management
according to a documented procedure.}
\Activity{2}{A software life cycle with predefined stages of
manageable size is identified or defined.}
% Looking at several, eg. SEL
\Activity{3}{The project's software development plan is developed
according to a documented procedure.}
\Activity{3}{The plan for the software project is documented.}
\Activity{3}{Software work products that are needed to establish
and maintain control of the software project are identified.}
\Activity{1}{Estimates for the size of the software work products (or
changes to the size of software work products) are derived according
to a documented procedure.}
\Activity{1}{Estimates for software project's effort and costs are
derived according to a documented procedure.}
\Activity{1}{Estimates for the project's critical computer
resources are derived according to a documented procedure.}
\Activity{2}{The project's software schedule is derived according
to a documented procedure.}
\Activity{0}{The software risks associated with the cost,
resource, schedule, and technical aspects of the project are
identified, assessed, and documented.}
\Activity{1}{Plans for the project's software engineering
facilities and support tools are prepared.}
\Activity{3}{Software planning data are recorded.}
% Website, Denise.
\end{KPAActivity}

\begin{KPARate}
\Approach{3}{Courses taken by several team members: 
             {\em Mastering Technical Projects}, Erika Jones \&
             Associates, Inc.;  
             {\em Mastering Projects Workshop}, True North pgs, Inc.;
             Staff for Project Management support}
\Deployment{2}{Project planned as part of training course., MS-Project
             plans (), project web site}
\Results{1}{}
\end{KPARate}

\newpage
\section{Software Project Tracking and Oversight}

\begin{KPAActivity}
\Activity{3}{A documented software development plan is used for
tracking the software activities and communicating status.}
% Project plans
\Activity{1}{The project's software development plan is revised
according to a documented procedure.}
\Activity{1}{Software project commitments and changes to
commitments made to individuals and groups external to the
organization are reviewed with senior management according to a
documented procedure.}
\Activity{3}{Approved changes to commitments that affect the
software project are communicated to the members of the software
engineering group and other software-related groups.}
\Activity{0}{The size of the software work products (or size of the
changes to the software work products) are tracked, and corrective
actions taken as necessary.}
\Activity{3}{The project's software effort and costs are tracked,
and corrective actions are taken as necessary.}
\Activity{3}{The project's critical computer resources are tracked,
and corrective actions are taken as necessary.}
\Activity{3}{The project's software schedule is tracked, and
corrective actions are taken as necessary.}
\Activity{3}{Software engineering technical activities are tracked,
and corrective actions are taken as necessary.}
\Activity{1}{The software risks associated with cost, resource,
schedule, and technical aspects of the project are tracked.}
\Activity{1}{Actual measurement data and replanning data for the
software project are recorded.}
\Activity{2}{The software engineering group conducts periodic
internal reviews to track technical progress, plans, performance, and
issues against the software development plan.}
\Activity{2}{Formal reviews to address the accomplishments and
results of the software project are conducted at selected project
milestones according to a documented procedure.} 
\end{KPAActivity}

\begin{KPARate}
\Approach{1}{MS-Project plans(), project web site}
\Deployment{1}{}
\Results{0}{}
\end{KPARate}

\newpage
\section{Software Subcontract Management}
% Our software subcontractors:
%	Geoff Furnish
%	SPRNG
%	POOMA
%	CIC-19

\begin{KPAActivity}
\Activity{0}{The work to be subcontracted is defined and planned
according to a documented procedure.}
\Activity{0}{The software subcontractor is selected , based on an
evaluation of the subcontractor bidders' ability to perform the work,
according to a documented procedure.}
\Activity{5}{The contractual agreement between the prime contractor
and the software subcontractor is used as the basis for managing the
subcontract.}
% Geoff's contract, SPRNG Contract  
\Activity{0}{A documented subcontractor's software development plan is
reviewed and approved by the prime contractor.}
\Activity{0}{A documented and approved subcontractor's software
development plan is used for tracking the software activities and
communication status.}
\Activity{1}{Changes to the software subcontractor's statement of
work, subcontract terms and conditions, and other commitments are
resolved according to a documented procedure.}
\Activity{3}{The prime contractor's management conducts periodic
status/coordination reviews with the software subcontractor's
management.}
\Activity{5}{Periodic technical reviews and interchanges are held with
the software subcontractor.}
% Monthly progress reports from Furnish ala MPW pg 5-20, PIP pg 6-34
\Activity{1}{Formal reviews to address the subcontractor's software
engineering accomplishments and results are conducted at selected
milestones according to a documented procedure.}
\Activity{0}{The prime contractor's software quality assurance group
monitors the subcontractor's software quality assurance activities
according to a documented procedure.}
\Activity{0}{The prime contractor's software configuration management
group monitors the subcontractor's activities for software
configuration management according to a documented procedure.}
\Activity{3}{The prime contractor conducts acceptance testing as part
of the delivery of the subcontractor's software products according to
a documented procedure.}
\Activity{4}{The software subcontractor's performance is evaluated on
a periodic basis, and the evaluation is reviewed with the
subcontractor.}
% Review of Furnish's contract.
\end{KPAActivity}

\begin{KPARate}
\Approach{2}{Statement of work contracts, monthly reports.}
\Deployment{2}{}
\Results{1}{}
\end{KPARate}

\newpage
\section{Software Quality Assurance}

\begin{KPAActivity}
\Activity{2}{An SQA plan is prepared for the software project
according to a documented procedure.}
\Activity{5}{The SQA group's activities are performed in accordance
with the SQA plan.}
% Meyer, Lakos, ATM, Test Problems 
\Activity{5}{The SQA group participates in the preparation and review
of the project's software development plan, standards, and
procedures.}
% Meyer, Lakos, ATM, Test Problems 
\Activity{4}{The SQA group reviews the software engineering activities
to verify compliance.}
\Activity{6}{The SQA group audits designated software work products to
verify compliance.}
% Meyer, Lakos, ATM, Test Problems 
\Activity{6}{The SQA group periodically reports the results of its
activities to the software engineering group.}
% Meyer, Lakos, ATM, Test Problems 
\Activity{4}{Deviations identified in the software activities and
software work products are documented and handled according to a
documented procedure.}
\Activity{2}{The SQA group conducts periodic reviews of its activities
and findings with the customer's SQA personnel, as appropriate.}
\end{KPAActivity}

\begin{KPARate}
\Approach{4}{SQA personnel sought by Management, Nightly regression
             tests run (dejagnu), Failure tracking software installed
             (GNATS) , Standard test cases established, Levelized
             Design and Design by Contract used}
\Deployment{4}{Tools consistently used}
\Results{4}{Faults caught early, Few major faults in testing}
\end{KPARate}

\newpage
\section{Software Configuration Management}

\begin{KPAActivity}
\Activity{4}{A SCM plan is prepared for each software project
according to a documented procedure.}
\Activity{4}{A documented and approved SCM plan is used as the basis
for performing the SCM activities.}
\Activity{6}{A configuration management library system is established
as a repository for the software baseline.}
% CVS
\Activity{4}{The software work products to be placed under
configuration management are identified.}
\Activity{3}{Change requests and problem reports for all configuration
items/units are initiated, recorded, reviewed, approved, and tracked
according to a documented procedure.}
\Activity{2}{Changes to baselines are controlled according to a
documented procedure.}
\Activity{1}{Products from the software baseline library are created
and their release is controlled according to a documented procedure.}
\Activity{1}{The status of configuration items/units is recorded
according to a documented procedure.}
\Activity{5}{Standard reports documenting the SCM activities and the
contents for the software baseline are developed and made available to
affected groups and individuals.}
% Draco Build document
\Activity{1}{Software baseline audits are conducted according to a
documented procedure.}
\end{KPAActivity}

\begin{KPARate}
\Approach{6}{Requirements, plans, documentation, and code under CVS.
             ChangeLogs used for change documentation.  All versions
             subject to nightly regression tests, tagged
             releases subject to standard test cases}
\Deployment{7}{Approach consistently used in project. Project members
             and Management automatically informed of configuration
             status.}
\Results{6}{Code state and history available.  Milestones identifiable
            in tagged releases.}
\end{KPARate}

\newpage
\section{Organization Process Focus}

\begin{KPAActivity}
\Activity{0}{The software process is assessed periodically, and action
plans are developed to address the assessment findings.}
\Activity{0}{The organization develops and maintains a plan for its
software process development and improvement activities.}
\Activity{0}{The organization's and projects' activities for
developing and improving their software processes are coordinated at
the organization level.}
\Activity{2}{The use of the organization's software process
database is coordinated at the organizational level.}
\Activity{0}{New processes, methods, and tools in limited use in
the organization are monitored, evaluated, and, where appropriate,
transfered to the other parts of the organization.}
\Activity{2}{Training for the organization's and projects' software
processes is coordinated across the organization.}
\Activity{0}{The groups involved in implementing the software
processes are informed of the organization's and projects' activities
for software process development and improvement.}
\end{KPAActivity}

\begin{KPARate}
\Approach{0}{}
\Deployment{0}{}
\Results{0}{}
\end{KPARate}

\newpage
\section{Organization Process Definition}

\begin{KPAActivity}
\Activity{0}{The organization's standard software process is
developed and maintained according to a documented procedure.}
\Activity{0}{The organization's standard software process is
documented according to established organization standards.}
\Activity{0}{Descriptions of software life cycles that are approved
for use by projects are documented and maintained.}
\Activity{0}{Guideline and criteria for the project's tailoring of
the organization's standard software process are developed and
maintained.}
\Activity{1}{The organization's software process database is
established and maintained.}
\Activity{0}{A library of software process-related documentation is
established and maintained.}
\end{KPAActivity}

\begin{KPARate}
\Approach{0}{}
\Deployment{0}{}
\Results{0}{}
\end{KPARate}

\newpage
\section{Training Program}

\begin{KPAActivity}
\Activity{0}{Each software project develops and maintains a training
plan that specifies its training needs.}
\Activity{0}{The organization's training plan is developed and
revised according to a documented procedure.}
\Activity{0}{The training for the organization is performed in
accordance with the organization's training plan.}
\Activity{0}{Training courses prepared at the organization level are
developed and maintained according to organization standards.}
\Activity{0}{A waiver procedure for required training is established
and used to determine whether individuals already possess the
knowledge and skills required to perform in their designated roles.}
\Activity{0}{Records of training are maintained.}
\end{KPAActivity}

\begin{KPARate}
\Approach{0}{}
\Deployment{0}{}
\Results{0}{}
\end{KPARate}

\begin{KPAComment}
Although we have no plan for tranning, there have been many formal and
informal courses associated with DRACO 99.  Formally, we have taken
{\em Mastering Projects Workshop} and {\em Managing Technical
Projects} courses; in the latter we used DRACO 99 as our course
project.  Informally, we have had team seminars on Design by Contract,
threads, and analytic test problems.
\end{KPAComment}
\newpage
\section{Integrated Software Management}

\begin{KPAActivity}
\Activity{1}{The project's defined software process is revised
according to a documented procedure.}
\Activity{1}{Each project's defined software process is revised
according to a documented procedure.}
\Activity{2}{The project's software development plan, which describes
the use of the project's defined software process, is developed and
revised according to a documented procedure.}
\Activity{3}{The software project is managed in accordance with the
project's defined software process.}
\Activity{1}{The organization's software process database is used for
software planning and estimating.}
\Activity{1}{The size of the software work products (or size of changes
to the software work products) is managed according to a documented
procedure.}
\Activity{1}{The project's software effort and costs are managed
according to a documented procedure.}
\Activity{1}{The project's critical computer resources are managed
according to a documented procedure.}
\Activity{1}{The critical dependencies and critical paths of the
project's software schedule are managed according to a documented
procedure.}
\Activity{1}{The project's software risks are identified, assessed,
documented, and managed according to a documented procedure.}
\Activity{1}{Reviews of the software project are periodically performed
to determine the actions needed to bring the software project's
performance and results in line with the current and projected needs
of the business, customer, and end users, as appropriate.}
\end{KPAActivity}

\begin{KPARate}
\Approach{1}{}
\Deployment{1}{}
\Results{1}{}
\end{KPARate}

\newpage
\section{Software Product Engineering}

\begin{KPAActivity}
\Activity{3}{Appropriate software engineering methods and tools are
integrated into the project's defined software process.}
\Activity{2}{The software requirements are developed, maintained,
documented, and verified by systematically analyzing the allocated
requirements according to the project's defined software process.}
\Activity{2}{The software design is developed, maintained, documented,
and verified according to the project's defined software process, to
accommodate the software requirements and to form the framework for
coding.}
\Activity{2}{The software code is developed, maintained, documented,
and verified according to the project's defined software process, to
implement the software requirements and software design.}
\Activity{3}{Software testing is performed according to the project's
defined software process.}
% LA-UR-98-5735
\Activity{3}{System and acceptance testing of the software are planned
and performed to demonstrate that the software satisfies its
requirements.}
\Activity{3}{The documentation that will be used to operate and
maintain the software is developed and maintained according to the
project's defined software process.}
\Activity{0}{Data on defects identified in peer reviews and testing are
collected and analyzed according to the project's defined software
process.}
\Activity{3}{Consistency is maintained across software work products,
including the software plans, process descriptions, allocated
requirements, software requirements, software design, code, test
plans, and test procedures.}
\end{KPAActivity}

\begin{KPARate}
\Approach{3}{}
\Deployment{3}{}
\Results{3}{}
\end{KPARate}

\newpage
\section{Intergroup Coordination}

\begin{KPAActivity}
\Activity{}{The software engineering group and the other engineering
groups participate with the customer and end users, as appropriate, to
establish the system requirements.}
\Activity{}{Representatives of the project's software engineering
group work with representatives of the other engineering groups to
monitor and coordinate technical activities and resolve technical
issues.}
\Activity{}{A documented plan is used to communicate intergroup
commitments and to coordinate and track the work performed.}
\Activity{}{Critical dependencies between engineering groups are
identified, negotiated, and tracked according to a documented
procedure.}
\Activity{}{Work products produced as input to other engineering groups
are reviewed by representatives of the receiving groups to ensure that
the work products meet their needs.}
\Activity{}{Intergroup issues not resolvable by the individual
representatives of the project engineering groups are handled
according to a documented procedure.}
\Activity{}{Representatives of the project engineering groups conduct
periodic technical reviews and interchanges.}
\end{KPAActivity}

\begin{KPARate}
\Approach{}{}
\Deployment{}{}
\Results{}{}
\end{KPARate}

\newpage
\section{Peer Reviews}

\begin{KPAActivity}
\Activity{}{Peer reviews are planned, and the plans are documented.}
\Activity{}{Peer reviews are performed according to a documented
procedure.}
\Activity{}{Data on the conduct and results of the peer reviews are
recorded.} 
\end{KPAActivity}

\begin{KPARate}
\Approach{}{}
\Deployment{}{}
\Results{}{}
\end{KPARate}

\newpage
% special section numbering for ASCI KPA
\renewcommand{\thesection}{ASCI KPA}
\section{Validation and Verification}

\begin{KPAActivity}
\Activity{}{V\&V implementation plan is known and followed by program
participants.}
\Activity{}{Data validation activities are documented according to the
V\&V implementation plan.}
\Activity{}{A documented test plan is followed at multiple phases of the
V\&V effort.}
\Activity{}{Validation requirements are documented.}
\Activity{}{Testing is performed in accordance with documented test
plans, test cases, and detailed test procedures.}
\Activity{}{Verification and validation is included as a software
process element that contains a set of closely related tasks that are
well defined and bounded.}
\Activity{}{Verification and validation costs are specifically
identified as such in software product cost estimates.}
\Activity{}{Peer reviews include verification and validation work
products.}
\end{KPAActivity}

\begin{KPARate}
\Approach{}{}
\Deployment{}{}
\Results{}{}
\end{KPARate}

\end{document}
