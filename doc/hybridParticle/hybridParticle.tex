%%---------------------------------------------------------------------------%%
%% hybridParticle.tex
%% Mike Buksas
%% $Id$
%%---------------------------------------------------------------------------%%
\documentclass[11pt]{nmemo}
\usepackage[centertags]{amsmath}
\usepackage{amssymb,amsthm,graphicx}
\usepackage[mathcal]{euscript}
\usepackage{tmadd,tmath}
\usepackage{cite}

%%---------------------------------------------------------------------------%%
%% DEFINE SPECIFIC ENVIRONMENTS HERE
%%---------------------------------------------------------------------------%%
%\newcommand{\elfit}{\ensuremath{\operatorname{Im}(-1/\epsilon(\vq,\omega)}}
%\msection{}-->section commands
%\tradem{}  -->add TM subscript to entry
%\ucatm{}   -->add trademark footnote about entry

%%---------------------------------------------------------------------------%%
%% BEGIN DOCUMENT
%%---------------------------------------------------------------------------%%
\begin{document}

%%---------------------------------------------------------------------------%%
%% OPTIONS FOR NOTE
%%---------------------------------------------------------------------------%%

\toms{Distribution}
%\toms{Joe Sixpak/XTM, MS B226}
\refno{CCS-4:01-33(U)}
\subject{Low-Weight IMC Particle Termination in Draco}

%-------NO CHANGES
\divisionname{Computer and Computational Sciences Division}
\groupname{CCS-4: Transport Methods Group}
\fromms{Mike Buksas/CCS-4 D409}
\phone{(505)667--7580/(505)665--5538}
\originator{mwb}
\typist{mwb}
\date{\today}
%-------NO CHANGES

%-------OPTIONS
%\reference{NPB Star Reimbursable Project}
%\thru{P. D. Soran, XTM, MS B226}
%\enc{list}      
%\attachments{list}
%\cy{list}
%\encas
%\attachmentas
%\attachmentsas 
%-------OPTIONS

%%---------------------------------------------------------------------------%%
%% DISTRIBUTION LIST
%%---------------------------------------------------------------------------%%

\distribution {
  T.E. Booth, X-5 MS F663 \\
  B.A. Clark, CCS-4 MS D409 \\
  T.M. Evans, CCS-4 MS D409 \\ 
  J.M. McGhee, CCS-4 MS D409 \\ 
  J.E. Morel, CCS-4 MS D409 \\ 
  M. Murillo, CCS-4 MS D409 \\
  G.L. Olson, CCS-4 MS D409 \\ 
  S. Pautz, CCS-4 MS D409 \\ 
  K.G. Thomson, CCS-4 MS D409 \\
  S.A. Turner, CCS-4 MS D409 \\ 
  T.J. Urbatsch, CCS-4 MS D409 \\ 
  J.S. Warsa, CCS-4 MS D409 \\
}

%%---------------------------------------------------------------------------%%
%% BEGIN NOTE
%%---------------------------------------------------------------------------%%

\opening

\section{Introduction}

In memo CCS-4:01-15(U), the authors propose the construction of a new
Particle type that would switch between implicit and analog absorption
behavior below an energy-weight threshold (an idea originally proposed
by Tom Booth, X-5). It was suggested that this hybrid particle would
eliminate a persistent difference between the computed material and
radiation energies apparent in thermal equilibrium problems. This
discrepancy was known to be caused by the ``full-clip Russian
roulette'' approach to terminating low weight particles, wherein all
particles below a fractional energy-weight threshold (1\% of their
original energy weight, typically) are harvested and their energy
deposited into the material.  Eliminating full-clip Russian roulette
will necessarily increase computational effort, because some particles
will survive longer, and increase variance since these particles will
be subject to further random events.  Nevertheless, full-clip Russian
roulette is biased because it delivers all of the particle energy to
the material, so replacing it with another particle termination method
is desirable, even at the expense of increased variance and
computation time. In this memo we describe the implementation of a
modified particle type with the hybrid absorption behavior and
demonstrate its effectiveness at closing the material/radiation
temperature gap at moderate cost.

\section{Implementation}

The hybrid Particle type was created by modifying the original
\texttt{Particle} type present in the IMC component directory of
Draco.  The \texttt{transport} member function of this new type was
modified to implement the new absorption behavior. The modified
\texttt{transport} method in the new class acts as before so long as
the fractional energy weight of the particle is above 0.01 (i.e. the
energy weight is greater than 1\% of its initial value). In this case,
the energy weight is decreased along the path of the particle to
represent implicit absorption. For particles that are below 0.01 of
their original energy weight, a cross section is computed for
absorption and the energy weight of the particle remains constant.

The new class was then added to the Draco/IMC library. A version of
\texttt{milagro\_xyz} was written that allowed the type of particle to
be specified when the code is run. (Both of these steps were greatly
facilitated by the use of a templated parameter throughout Draco and
Milagro to specify the particle type.) We tested the new particle type
by using it to compile and run \texttt{tstParticle} in the Draco
regression test suite. Furthermore, it produced results identical to
the original particle type on the pure streaming and pure scattering
problems in the test suite for \texttt{milagro\_xyz}.

\section{Results}

\subsection{Steady State Problem}

We test the efficacy of the hybrid particle on a steady-state problem
in which the material and radiation are in thermal equilibrium at a
common initial temperature of 0.89994 keV. We compute the average and
standard deviation of the material and radiation temperatures in a
single cell over ten different runs with different random seeds. The
data is collected at 0.1 shakes (100 steps at 0.001 shakes per step)
and the runs are performed with 1,000 particles.

\begin{center}
  \begin{tabular}{r|cc}
    ``Full-Clip'' Particle  &  Average   & STD \\ \hline
     Material Temperature:  &  0.902114  & $4.08194\times 10^{-4}$ \\
     Radiation Temperature: &  0.899231  & $8.83107\times 10^{-5}$ \\
     Difference:            &  0.002883  & \\
     \multicolumn{3}{c}{} \\
     Hybrid Particle        &  Average   & STD \\ \hline
     Material Temperature:  &  0.899494  & $4.78574\times 10^{-4}$ \\
     Radiation Temperature: &  0.899961  & $1.81940\times 10^{-4}$ \\
     Difference:            & -0.000467  & \\
  \end{tabular}
\end{center}

We notice that the discrepancy in the material and radiation
temperatures is smaller by a factor of six and is no longer in the
characteristic direction of greater material temperature. As expected,
an increase in the variance is also observed. The increase in variance
is sufficiently small that the merit of the new absorption behavior is
not substantially reduced.  

Because of optimizations made in the transport and other methods of
the new particle class, the execution was faster for the hybrid
particle than the original version (92.11 versus 100.93 seconds, on
average) despite requiring more computation.

\subsection{Marshak wave problem}

The Marshak wave problem is a diffusive radiative transfer problem in
which the material and radiation are in thermal equilibrium.  We use
it here to demonstrate that the original and hybrid particle types
give qualitatively similar results. The particular problem is
marshak2b as implemented in problem \texttt{fulltp01} in the Milagro
regression test suite. After 10 shakes (10,000 time-steps with 10,000
particles) the analytic solution and the solutions generated using
each particle type and the same initial random seed are as pictured in
Figure~\ref{fig:marshak2b}.

\vspace{1em}
\begin{figure}[h]
  \centerline{\includegraphics[height=3.0in]{plot.eps}}
  \caption{Analytic solution, and Monte Carlo solutions using Analog
    and Hybrid Particles to the Marshak2b problem}
  \label{fig:marshak2b}
\end{figure}

\section{Conclusions}

We have demonstrated that replacing the full-clip Russian roulette
termination of low weight particles with analog absorption effectively
eliminates the material/energy temperature bias in equilibrium
problems. The elimination of the bias is obtained at the cost of a
slight increase in the variance and amount of computation.  The time
required for the additional computation was more than made up for by
other improvements in the efficiency of the execution.  Furthermore,
we expect that this method of particle termination will reduce the
sensitivity of the computation to the low-weight cut off value since
the absorption of the particle is unbiased both above and below of the
cutoff.

\newpage

\closing
\end{document}

%%---------------------------------------------------------------------------%%
%% end of hybridParticle.tex
%%---------------------------------------------------------------------------%%
