%  ========================================================================  %
% 
% 	Author:	Mark G. Gray
% 		Los Alamos National Laboratory
% 	Date:	Wed Apr  5 09:42:35 MDT 2000
% 
% 	Copyright (c) 2000 U. S. Department of Energy. All rights reserved.
% 
%  	$Id$
% 
%  ========================================================================  %

\documentclass[11pt]{nmemo}
\usepackage{amssymb,amsthm,graphicx}

%%---------------------------------------------------------------------------%%
%% BEGIN DOCUMENT
%%---------------------------------------------------------------------------%%

\newcommand{\fninety}{\texttt{Fortran~90}}
\newcommand{\cpp}{\texttt{C++}}
\newcommand{\withfninety}{\texttt{--with-f90}}
\newcommand{\langfninety}{\texttt{AC\_LANG\_F90}}
\newcommand{\progfninety}{\texttt{AC\_PROG\_F90}}
\newcommand{\requiregmfour}{\texttt{AC\_REQUIRE\_GM4}}
\newcommand{\proggmfour}{\texttt{AC\_PROG\_GM4}}

\begin{document}

%%---------------------------------------------------------------------------%%
%% OPTIONS FOR NOTE
%%---------------------------------------------------------------------------%%

\toms{Distribution}
\refno{X6:MGG-00-XX (U)}
\subject{\fninety\ Build Support in Draco}
\divisionname{Applied Physics Division}
\groupname{X--6:Transport Methods Group}
\fromms{Mark G. Gray/X--6, MS D409}
\phone{(505)667--5341}
\originator{mgg}
\typist{mgg}
\date{\today}

%%---------------------------------------------------------------------------%%
%% DISTRIBUTION LIST
%%---------------------------------------------------------------------------%%

\distribution {
  J.E. Morel, XTM MS D409\\
  J.M. McGhee, XTM MS D409\\
  H.G. Hughes, XTM MS D409\\
  T.M. Evans, XTM MS D409\\
  M.G. Gray, XTM MS D409\\
  S.D. Pautz, XTM MS D409\\
  R.M. Roberts, XTM MS D409\\
  T.J. Urbatsch, XTM MS D409\\
  T.A. Wareing, XTM MS D409\\
  J.S. Warsa, XTM MS D409\\
  C.J. Gesh, XTM MS D409\\
  W.D. Hawkins, XTM MS D409\\
  B.T. Adams, XTM MS D409 \\
  M.L. Alme, XTM MS D409\\
  J.C. Gulick, XTM D409\\
  K.G. Thompson, XCI, MS F663\\
  R.B. Lowrie, XHM MS D413 
  }

%%---------------------------------------------------------------------------%%
%% BEGIN NOTE
%%---------------------------------------------------------------------------%%

\opening

% I want to tell John that...

I have added rudimentary \fninety\ support to the Draco build
system\cite{draco-build}.  The autoconf\cite{autoconf} macros in
\texttt{draco/config} now support the \withfninety\ argument, which
allows the user to specify the \fninety\ compiler to use; the
\langfninety\ and \progfninety\ macros, which specify \fninety\ as the
language used in compiler tests and test the selected compiler,
respectively; and the \proggmfour\ and \requiregmfour\ macros, which
find and require the GNU m4\cite{m4} preprocessor, respectively.
These features provide basic functionality for configuring a \fninety\
package.  They form the beginnings of full \fninety\ support in Draco,
provide a template for other language support in Draco, and suggest a
possible refactoring of the existing build macros.

\section{Background}

Draco was originally concieved as a collection of \cpp\ packages that
could be used to construct transport codes.  An unexpected bonus of
the Draco project was a robust build system, based on autoconf, that
could be used to configure any collection of \cpp\ code for various
platforms.  The simple, reusable build system required several
iterations of careful planning and hard work.

One feature missing from the build system was support for \fninety.
Although Draco will remain primarily a \cpp\ repository, \fninety\
interface packages, numerical packages, and vendor packages are
planned additions.  Further, just as the build system in Draco is
reused in other \cpp\ projects, a common build system that could be
reused by other \fninety\ projects is desirable.

I had written rudimentary \fninety\ support as part of the Zathras and
Centauri projects.  This support was based on the autoconf support
written for Dante by Randy Roberts.  With the advent of the new Linux
workstations and their new Fujitsu compiler I needed to at least
revisit my old \fninety\ autoconf macros.  This seemed to be an ideal
time to look into incorporating that \fninety\ support in Draco.

I examined Tom Evans work on the autoconf \cpp\ macros for the build
system, and found the design sound and highly ameanable to the
necessary \fninety\ additions.  This memo documents the addition of
several necessary build system features for \fninety\ support.

\newpage

\section{Introduction}

How should the Draco build system support \fninety?  Ideally the user
should see a set of macros that work just like autoconf's support for
other languages.  The solution should be backwards compatable;
existing Draco configuration files should continue to work.  In this
section I will examine how the Draco build system works for \cpp, the
problems involved with applying this approach to \fninety, and the
resolution of these problems.

How does the Draco build system support \cpp?  Autoconf's support for
\cpp\ works as follows:
\begin{enumerate} 
\item \cpp\ is made the default language used for testing by the macro
\texttt{AC\_LANG\_CPLUSPLUS}
\item The compiler availability and viability are tested by
the macro \texttt{AC\_PROG\_CXX}
\end{enumerate}
\cpp\ compiler tests can be made without knowledge of the specific
vendor because \cpp\ compilers are remarkably uniform from platform to
platform. \texttt{CC -g Conftest.cc} will produce a working \cpp\
program on virtually every UNIX platform in existance.

The Draco build system adds an additional step: A specific \cpp\
compiler is requested by the configure argument \texttt{--with-cxx},
and compiler specific flags are set by the macro
\texttt{AC\_DRACO\_ENV}.  This additional step requires that autoconf
know something about specific compilers; this database is kept in the
file \texttt{ac\_compilers.m4}.  The disadvantage of giving autoconf
this specific knowledge is that it must be maintained; the advantage
is that \texttt{AC\_PROG\_CXX} can check if the user requested flags
work, and the build system can use simpler makefile templates.

Unfortunately the basic autoconf method of supporting \cpp\ will not
work for \fninety\ because:
\begin{enumerate}
\item Autoconf is designed primarily for \texttt{C} and \cpp; it was
not designed to supporting other languages.
\item \fninety\ compilers show little consistency across platforms;
the compiler name, flags, default file extensions, and method of
handling modules varies greatly from product to product.
\item The assumptions autoconf makes because of item 1 and the
inconsistency of \fninety\ compilers stated in item 2 makes the
autoconf method of guessing compiler names, flags, etc. actually
dangerous.
\end{enumerate}

Fortunately, the Draco additions provide a solution; in order to add
\fninety\ support to the Draco build system: 
\begin{enumerate}
\item Autoconf macros must be written for \fninety\ support; some
native autoconf macros must be rewritten to redress autoconf's
\texttt{C}-centricness  
\item \fninety\ compiler information must be available in a database
to configure names, flags, extensions, and module information
\item a better guessing heuristic must be devised.
\end{enumerate}

In the spirit of the Draco build system, \fninety\ support aims at
simplifying the configuration for the user, at the expense of
considerable work on the part of the build system to hide the
underlying complexity.

\section{New Arguments}

The \fninety\ compiler can be specified by giving the \withfninety\ 
argument to configure.  The currently supported compilers are shown in
Table~\ref{tbl:compilers}.

\begin{table}[hb]
\hrulefill
\begin{center}
\caption{Supported \fninety\ compilers}\label{tbl:compilers}
\begin{tabular}{l|l|l|l}
Compiler        & Executable & Target  & Target \\
(\withfninety=) & File       & OS      & Platform \\ \hline
Fujitsu         & f90        & Linux   & ix86 \\
XL	        & xlf90      & AIX     & RS6000 \\
WorkShop        & f90	     & Solaris & Sparc \\
Cray            & f90        & UNICOS  & Y-MP \\
MIPS            & f90        & IRIX    & SGI  
\end{tabular}
\end{center}
\hrulefill
\end{table}
The \langfninety\ macro must be present in \texttt{configure.in} for
the \withfninety\ argument to be processed.  If \withfninety\ is not
given or given without any arguments, the compiler is set based on the
target.  The compiler is verified by running \texttt{\$F90 --version}
(or its equivalent) and checking the output for the product name.

A new compiler can be added by putting a database entry for it in
\texttt{ac\_compiler.m4}, adding it to the case statement in
\texttt{ac\_dracoenv.m4}, and adding it to the argument option string
in \texttt{ac\_dracoarg.m4}.  The \withfninety\ argument is based on
the \texttt{--with-cxx} argument in \texttt{ac\_dracoarg.m4}.

\section{New Macros}

Four new macros have been added for \fninety\ support:
\begin{description}
\item[\langfninety:] set the compile and link test invocation
using \texttt{F90} and the extension \texttt{F90EXT}.  This macro is
required in \texttt{configure.in} for \fninety\ usage.  It is based on
the \texttt{AC\_LANG\_C} and \texttt{AC\_LANG\_CPLUSPLUS} macros in
\texttt{acgeneral.m4}.

\item[\proggmfour:] set the output variable \texttt{GM4} to a command
that runs the GNU m4 preprocessor.  Looks for \texttt{gm4} and then
\texttt{m4}, and verifies the gnu version by running
\texttt{\${GM4}~--version}.  It is based on the \texttt{AC\_PROG\_CPP}
and \texttt{AC\_PROG\_CPPCXX} macros in \texttt{acspecific.m4}.

\item[\requiregmfour:] ensure that \texttt{GM4} has been found by
calling \proggmfour\ if it hasn't already been called.  It is based on
the \texttt{AC\_REQUIRE\_CPP} macro in \texttt{acspecific.m4}.

\item[\progfninety:] determine the \fninety\ compiler to use and the
appropriate extensions for free format source code.  Set the
following output variables:
\begin{description}
\item[\texttt{F90}:] name of a working \fninety\ compiler.
\item[\texttt{F90FLAGS}:] vendor specific default flags for the
\fninety\ compiler.  Changes appropriately according to the
\texttt{--enable-debug} and \texttt{--with-opt} configure arguments.
\item[\texttt{F90EXT}:] vendor specific extension of \fninety\ input files.
\item[\texttt{F90FREE}:] vendor specific flag to specify free form
source input. 
\item[\texttt{F90FIXED}:] vendor specific flag to specify fixed form
source input. 
\item[\texttt{MODNAME}:] vendor specific module file name.  Module
files containing interface information are generated by the compiler
when a file containing a module is found.  Module files may be named
after the file containing the module, or after the module name, in
either all caps, all lowercase, or mixed case.  Use a compile test to
determine which is the case.   
\item[\texttt{MODSUFFIX}:] vendor specific module extension name.
Module files may have extension \texttt{mod}, \texttt{M}, \texttt{o}
(i.e. be part of the object file), or something else.  Use a compile
test to determine which is the case.
\item[\texttt{MODFLAG}:] vendor specific module include directory
flag, typically \texttt{-I} or \texttt{-M}.
\end{description}
It is based on the \texttt{AC\_PROG\_CC} and \texttt{AC\_PROG\_CXX}
macros in \texttt{acspecific.m4}.
\end{description}

These macros have only been tested on a limited number of machines.
\progfninety\ can fail due to vendor non-standard file extentions or
incorrect free/fixed source defaults.  Output variables are correctly
set for only a few known targets.  As with other autoconf macros, any
file named [Cc]onftest* will be overwritten!

\section{Example Fortran 90 Configuration}

\begin{figure}[hbt]
\hrulefill
\begin{verbatim}
AC_INIT(acf90)                  dnl process args; verify src dir
AC_CONFIG_AUX_DIR(../../config) dnl config.sub, config.guess, install-sh

AC_LANG_F90                     dnl set language to Fortran 90
AC_DRACO_ENV                    dnl configure Draco build environment
AC_PROG_F90                     dnl test Fortran 90 compiler
AC_PROG_GM4                     dnl Find and verify GNU m4

AC_OUTPUT(Makefile)             dnl create output files
\end{verbatim}
\caption{Sample \texttt{configure.in}.}\label{fig:configure}
\hrulefill
\end{figure}

Figure~\ref{fig:configure} shows a sample \texttt{configure.in} using
the \fninety\ macros.  The preamble lines initialize autoconf, give it
a file to check to verify that it is using the proper source
directory, and specify where to find the common scripts needed by
configure.  

These files and the aclocal files autoconf uses to create
\texttt{configure} from \texttt{configure.in} are kept in the
\texttt{config/} directory which is a peer of \texttt{src/}.
Figure~\ref{fig:zathras} shows the CVS module entry for Zathras, which
instructs \texttt{cvs checkout Zathras} to creates the directory
\texttt{zathras}, populate it with the contents of
\texttt{\$CVSROOT/zathras}, and rename \texttt{\$CVSROOT/draco/config}
as \texttt{zathras/config}.  This is the standard way to import the
Draco build system in external project directories.
\begin{figure}[hbt]
\hrulefill
\begin{verbatim}
# Zathras and friends:

Zathras         -a zathras zathras_config
zathras_config  -d zathras/config draco/config
\end{verbatim}
\caption{CVS module entry for Zathras.  The configure files for the
Draco build system located in \texttt{\$CVSROOT/draco/config} are put
into the \texttt{zathras/config} directory on
checkout.}\label{fig:zathras}
\hrulefill
\end{figure}

The body of Figure~\ref{fig:configure} contains the macros for
\fninety\ support.  \langfninety\ instructs configure to use \fninety\
for subsequent compiler tests.  \texttt{AC\_DRACO\_ENV} processes
configure arguments, including the selection of \fninety\ compiler
based on either the \withfninety\ argument or the target platform and
flags based on the \texttt{--enable-debug} and \texttt{--with-opt}
arguments.  \progfninety\ test the compiler to make sure it can
compile a null program, and discovers how to handles modules.

The last line processes \texttt{Makefile.in} to produce a
\texttt{Makefile} with the appropriate variable substitutions made.

Figure~\ref{fig:makefile} shows a test \texttt{Makefile.in} that
illustrates the use of the \fninety\ variables.  
\begin{figure}[phbt]
\hrulefill
\begin{verbatim}
# Macro Definitions

MODSUFFIX       =       @MODSUFFIX@
MODNAME         =       @MODNAME@
MODFLAG         =       @MODFLAG@

F90             =       @F90@
F90FLAGS        =       @F90FLAGS@
F90EXT          =       @F90EXT@
F90FREE         =       @F90FREE@
F90FIXED        =       @F90FIXED@
GM4             =       @GM4@

# Dependency and commands

all:
        @echo "F90       = ${F90}"
        @echo "F90FLAGS  = ${F90FLAGS}"
        @echo "F90EXT    = ${F90EXT}"
        @echo "F90FREE   = ${F90FREE}"
        @echo "F90FIXED  = ${F90FIXED}"
        @echo "MODNAME   = ${MODNAME}"
        @echo "MODSUFFIX = ${MODSUFFIX}"
        @echo "MODFLAG   = ${MODFLAG}"
        @echo "GM4       = ${GM4}"
\end{verbatim}%$
\caption{Sample Makefile.in.  The macro definitions (rhs) are expanded
by the configure script resulting from processing the configure.in
given in Figure~\ref{fig:configure}.  The single target prints the
resulting definitions.}\label{fig:makefile} \hrulefill
\end{figure}

Running autoconf on the \texttt{configure.in} file from
Figure~\ref{fig:configure} and running the resulting
\texttt{configure} file in a directory containing the test
\texttt{Makefile.in} produces a {Makefile} which, when executed,
produces the output in Figure~\ref{fig:make}.
\begin{figure}[phbt]
\hrulefill
\begin{verbatim}
azathoth $ make
F90       = f90
F90FLAGS  = -O0 -X9 -Am
F90EXT    = f90
F90FREE   = -Free
F90FIXED  = -Fixed
MODNAME   = modname
MODSUFFIX = mod
MODFLAG   = -I
GM4       = m4
\end{verbatim}%$
\caption{Result of sample make on azathoth (a Linux
machine).  The compiler is named f90.  The flags specify level 0
optimization, \fninety\ standard, and module generation.  The proper
file extension is .f90.  Free and fixed source are specified with
-Free and -Fixed, respectively.  Module files are named after the
module, with extension .mod.  Module include directories are specified
with -I.  GNU m4 was found, under the name m4.}\label{fig:make}
\hrulefill
\end{figure}

\section{Conclusion}

I have written autoconf macros that provide rudimentary \fninety\
support for the Draco build system.  For \cpp\ clients, nothing in the
build system is affected.  For \fninety\ clients (defined by the use of
\langfninety\ in \texttt{configure.in}) these macros find an
appropriate \fninety\ compiler, verify its operation, and set the
flags needed for its use as a makefile command.

Much work remains.  A standard \texttt{Makefile.in}
for \fninety\ source directories, and a tool for ordering dependencies
within and perhaps across directories is needed before Draco truely
supports \fninety\ builds.  

The \langfninety\ and \progfninety\ macros provide templates for
adding other language support in Draco.  In theory, with a minor
modification to \texttt{AC\_LANG\_RESTORE}, a single
\texttt{configure} file could select and configure for several
languages, permitting the use of mixed language directories in Draco.
However, this feature has not been tested; current plans are for all
Draco components (directories) to be unilingual.

Future language support, including improved support for \fninety,
could be made easier by refactoring parts of the existing build
support for \cpp.  This could simplify build macros, make the
build system less \cpp-centric, and improve the chance of having one
template \texttt{Makefile.in} serve all.

\bibliographystyle{plain}
\bibliography{../bib/draco}
\clearpage
\closing

\end{document}

%  ========================================================================  %
