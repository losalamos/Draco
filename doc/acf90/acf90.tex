%  ========================================================================  %
% 
% 	Author:	Mark G. Gray
% 		Los Alamos National Laboratory
% 	Date:	Wed Apr  5 09:42:35 MDT 2000
% 
% 	Copyright (c) 2000 U. S. Department of Energy. All rights reserved.
% 
%  	$Id$
% 
%  ========================================================================  %

\documentclass[11pt]{nmemo}
\usepackage[centertags]{amsmath}
\usepackage{amssymb,amsthm,graphicx}
\usepackage[mathcal]{euscript}
\usepackage{tmadd,tmath}
\usepackage{c++}
\usepackage{cite}

%%---------------------------------------------------------------------------%%
%% BEGIN DOCUMENT
%%---------------------------------------------------------------------------%%

\begin{document}

%%---------------------------------------------------------------------------%%
%% OPTIONS FOR NOTE
%%---------------------------------------------------------------------------%%

\toms{Distribution}
%\toms{Joe Sixpak/X--TM, MS D409}
\refno{X6:MGG-00-XX (U)}
\subject{Fortran 90 Build Support in Draco}

%-------NO CHANGES
\divisionname{Applied Physics Division}
\groupname{X--6:Transport Methods Group}
\fromms{Mark G. Gray/X--6, MS D409}
\phone{(505)667--5341}
\originator{mgg}
\typist{mgg}
\date{\today}
%-------NO CHANGES

%-------OPTIONS
%\reference{NPB Star Reimbursable Project}
%\thru{P. D. Soran, XTM, MS B226}
%\enc{list}      
%\attachments{list}
%\cy{list}
%\encas
%\attachmentas
%\attachmentsas 
%-------OPTIONS

%%---------------------------------------------------------------------------%%
%% DISTRIBUTION LIST
%%---------------------------------------------------------------------------%%

\distribution {
  J.E. Morel, XTM MS D409\\
  J.M. McGhee, XTM MS D409\\
  H.G. Hughes, XTM MS D409\\
  T.M. Evans, XTM MS D409\\
  M.G. Gray, XTM MS D409\\
  S.D. Pautz, XTM MS D409\\
  R.M. Roberts, XTM MS D409\\
  T.J. Urbatsch, XTM MS D409\\
  T.A. Wareing, XTM MS D409\\
  J.S. Warsa, XTM MS D409\\
  C.J. Gesh, XTM MS D409\\
  W.D. Hawkins, XTM MS D409\\
  B.T. Adams, XTM MS D409 \\
  M.L. Alme, XTM MS D409\\
  J.C. Gulick, XTM D409\\
  K.G. Thompson, XCI, MS F663\\
  R.B. Lowrie, XHM MS D413 
  }

%%---------------------------------------------------------------------------%%
%% BEGIN NOTE
%%---------------------------------------------------------------------------%%

\opening

% I want to tell John that...

I have added basic Fortran 90 compiler support to the Draco build
system.  The autoconf macros in draco/config now support the
--with-f90 argument with allows the user to specify the Fortran 90
compiler to use.  Additionally, the AC\_LANG\_F90, and AC\_PROG\_F90
macros can be used in configure.in to specify Fortran 90 as the
language use in compiler tests and to test the selected compiler,
respectively.  Finally, the AC\_PROG\_GM4 and AC\_REQUIRE\_GM4 macros
can be used in configure.in to find and require the GNU m4
preprocessor, respectively.  This argument and these macros provide
the necessary functionality for configuring a Fortran 90 package.

\section{Background}

Draco was originally concieved as a collection of C++ packages that
could be used to construct transport codes.  An unexpected bonus of
the Draco project was a robust build system that could be used to
configure a package of C++ code for various platforms.  Typical of
such reuse, the build system required much careful planning and hard
work to create, but provided a system that is easy to use.

One feature missing from the build system was support for Fortran 90.
Although Draco will remain primarily a C++ repository, Fortran 90
interface packages, numerical packages, and vendor packages are
planned additions.  Further, just as the build system in Draco is used
in several C++ projects, a common build system which could
be used by several Fortran 90 projects is desirable.   This memo
documents the addition of several necessary build system features for
Fortran 90 support.

\newpage

\section{Introduction}

Configuring for Fortran 90 is a complicated business.  In the spirit
of the Draco build system, Fortran 90 support aims at simplifying
the configuration for the user, at the expense of considerable work on
the part of the build system.

\section{New Arguments}

The Fortran 90 compiler can be specified by giving the --with-f90
argument to configure.  The currently supported compilers are shown in
Table~\ref{tbl:compilers}.

\begin{table}[hb]
\caption{Supported Fortran 90 compilers}\label{tbl:compilers}
\begin{tabular}{l|l|l|l}
Compiler & Executable & Target OS & Target Platform \\ \hline
Fujitsu  & f90        & Linux     & ix86 \\
XL	 & xlf90      & AUX       & RS6000 \\
WorkShop & f90	      & Solaris	  & Sparc \\
Cray     & f90        & UNICOS    & Y-MP \\
MIPS     & f90        & IRIX      & SGI  
\end{tabular}
\end{table}
If --with-f90 is given without any arguments, an attempt to guess the
compiler is made by the configure script based on the target.

\section{New Macros}

AC\_LANG\_F90 sets up the compile and link test invocation using
F90 and the extension F90EXT.

AC\_PROG\_GM4 sets the output variable GM4 to a command that runs 
the GNU m4 preprocessor.  Looks for gm4 and then m4, and verifies 
the gnu version by running \${GM4} --version

AC\_REQUIRE\_GM4 ensures that GM4 has been found

AC\_PROG\_F90 determines the Fortran 90 compiler to use and the
appropriate extension for free format source code.  If F90
is not already set in the environment, check for `f90'.  Try
the compiler with the extensions `f90', `F', `F90', and `f'.
Set the output variable F90 to the name of the compiler found
and F90EXT to the name of the free format source extension.
If the output variable F90FLAGS was not already set, set it to
`-g'.  If the output variables F90FREE and F90FIXED were not
already set, try to guess their values from the target.  If
the output variable MODFLAG was not already set, try to guess
its value from the target.

BUGS

These macros have only been tested on a limited number of
machines.   AC\_PROG\_F90 can fail due to vendor non-standard
file extentions or incorrect free/fixed source defaults.
F90FREE and F90FIXED correctly set for only a few known
targets.  As with other autoconf macros, any file
named [Cc]onftest* will be overwritten!

\section{F90 configure.in}

Figure~\ref{fig:configure} shows a sample configure.in using the
Fortran 90 macros.  Line 1 processes command line arguments and finds
the source file directory.  Line 2 specifies where to find
\tt{install-sh}, \tt{config.sub}, and \tt{config.guess}.  Line 4 sets
the language to Fortran 90, with appropriate compile and link tests.
Line 5 sets up the draco build environment.  Line 6 tests the Fortran
90 compiler and discovers the module naming convention.  Line 7 finds
and verifies the GNU m4 preprocessor.  Line 9 creates the output files.

\begin{figure}[hb]
\begin{verbatim}
AC_INIT(Makefile.in)		# 1
AC_CONFIG_AUX_DIR(../../config) # 2
                                # 3
AC_LANG_F90                     # 4
AC_DRACO_ENV                    # 5
AC_PROG_F90                     # 6
AC_PROG_GM4                     # 7
                                # 8
AC_OUTPUT(Makefile)             # 9
\end{verbatim}
\caption{Sample configure.in}\label{fig:configure}
\end{figure}

\begin{figure}
\begin{verbatim}
MODSUFFIX       =       @MODSUFFIX@
MODNAME         =       @MODNAME@
MODFLAG         =       @MODFLAG@

F90             =       @F90@
F90FLAGS        =       @F90FLAGS@
F90EXT          =       @F90EXT@
F90FREE         =       @F90FREE@
F90FIXED        =       @F90FIXED@
GM4             =       @GM4@

SRCS            =       Makefile.in aclocal.m4 configure configure.in \
                        install-sh config.sub config.guess

all:
        @echo "F90       = ${F90}"
        @echo "F90FLAGS  = ${F90FLAGS}"
        @echo "F90EXT    = ${F90EXT}"
        @echo "F90FREE   = ${F90FREE}"
        @echo "F90FIXED  = ${F90FIXED}"
        @echo "MODNAME   = ${MODNAME}"
        @echo "MODSUFFIX = ${MODSUFFIX}"
        @echo "MODFLAG   = ${MODFLAG}"
        @echo "GM4       = ${GM4}"
\end{verbatim}%$
\caption{Sample Makefile.in}\label{fig:makefile}
\end{figure}

\begin{figure}
\begin{verbatim}
azathoth $ make
F90       = f90
F90FLAGS  = -O0 -X9 -Am
F90EXT    = f90
F90FREE   = -Free
F90FIXED  = -Fixed
MODNAME   = modname
MODSUFFIX = mod
MODFLAG   = -I
GM4      = m4
\end{verbatim}%$
\caption{Result of sample make on azathoth (a Linux
machine)}\label{fig:make}
\end{figure}
\newpage
\closing

\end{document}

%  ========================================================================  %
