%%---------------------------------------------------------------------------%%
%% matFactory.tex
%% Randy M. Roberts
%% $Id$
%%---------------------------------------------------------------------------%%
\documentclass[11pt]{nmemo}
\usepackage[centertags]{amsmath}
\usepackage{amssymb,amsthm,graphicx}
\usepackage[mathcal]{euscript}
\usepackage{tmadd,tmath}
\usepackage{cite}
\usepackage{c++}
\usepackage{fancycodes}
\usepackage[dvips]{color}

%%---------------------------------------------------------------------------%%
%% DEFINE SPECIFIC ENVIRONMENTS HERE
%%---------------------------------------------------------------------------%%
%\newcommand{\elfit}{\ensuremath{\operatorname{Im}(-1/\epsilon(\vq,\omega)}}
%\msection{}-->section commands
%\tradem{}  -->add TM subscript to entry
%\ucatm{}   -->add trademark footnote about entry

\definecolor{codecolor}{rgb}{0,0,1}
\definecolor{comcolor}{rgb}{1,0,0}
\newcommand{\cxxcom}{\color{comcolor}}
\newcommand{\cxxcode}{\color{codecolor}}

\newcommand{\code}[1]{\textcolor{codecolor}{#1}}

\newcommand{\tmpl}[1]{$<$#1$>$}

%%---------------------------------------------------------------------------%%
%% BEGIN DOCUMENT
%%---------------------------------------------------------------------------%%
\begin{document}

%%---------------------------------------------------------------------------%%
%% OPTIONS FOR NOTE
%%---------------------------------------------------------------------------%%

\toms{Distribution}

\refno{XTM-99-??? (U)}
\subject{Matrix Factory Traits}

%-------NO CHANGES
\divisionname{Applied Theoretical \& Computational Physics Div.}
\groupname{X-TM:Transport Methods Group}
\fromms{Randy M. Roberts/XTM D409}
\phone{(505)665--4285}
\originator{rmr}
\typist{rmr}
\date{\today}
%-------NO CHANGES

%-------OPTIONS
%\reference{NPB Star Reimbursable Project}
%\thru{P. D. Soran, XTM, MS B226}
%\enc{list}      
%\attachments{list}
%\cy{list}
%\encas
%\attachmentas
%\attachmentsas 
%-------OPTIONS

%%---------------------------------------------------------------------------%%
%% DISTRIBUTION LIST
%%---------------------------------------------------------------------------%%

\distribution {}

%%---------------------------------------------------------------------------%%
%% BEGIN NOTE
%%---------------------------------------------------------------------------%%

\opening

\section{Introduction}

One design requirement for the diffusion solver used by the
Solon P1 3T package is the ability to 
incorporate disparate linear algebra packages.
A complication resulting from this requirement arises from each linear
algebra package having different matrix representations.
The diffusion solver
must have a mechanism to generate matrices of differing representations
using a common mechanism, the \code{MatrixFactoryTraits}
mechanism.\cite{Austern99}\cite{Myers96}

These differing matrix representations result from requirements for
the efficient storage and manipulation of matrix, depending on the structure
of the linear system, and the solution algorithm.
For example, the linear system resulting
from a one-dimensional three-point standard
diffusion stencil may best be represented by a tri-diagonal matrix.
The solution of this system should
be specialized specifically for a tri-diagonal
system.\cite{Press92}

The diffusion solvers under Draco cannot know about every possible matrix
representation for every possible linear algebra package.
With a common mechanism to create a matrix of any desired representation
the diffusion solver can be written independent of the linear algebra
package.
When a new linear algebra package is incorporated into Draco it will be part
of the incorporation task to specialize a \code{MatrixFactoryTraits} class
for any new matrix representations.
The details of this class will be discussed below.

\section{Overview}

Draco would support its own set of matrix representations to be used during
the generation of linear systems.
These will probably not be the most efficient representations for matrix
computations, but will be efficient for generating linear systems of various
structures.
For example, we may have a dynamic compressed row storage representation,
which is very efficient for the insertion of new elements into a linear
system with an unknown sparsity pattern.
Another Draco matrix may be most efficient for representing a three-dimensional,
nine-point structured diffusion system.

The \code{MatrixFactoryTraits} mechanism would facilitate the creation
of a matrix of unforeseen representation, specified by the linear algebra
package, from one or more of the Draco matrix
representations.

\section{How the \code{MatrixFactoryTraits} mechanism is used.}

The \code{MatrixFactoryTraits} mechanism relies on the
\code{MatrixFactoryTraits} class specialized for the linear algebra
package's matrix representations.
This class provides the \code{create} static class method, which takes
a constant reference to a Draco specified matrix representation, and creates
a pointer to the linear algebra package's matrix.

\begin{ttfamily}
\cxxcode
  MyDracoMatrix myDracoMatrix(...);
\end{ttfamily}

The Draco diffusion solver would generate a Draco-defined matrix that most
efficiently represents the linear system that the diffusion solver generates.
The diffusion solver then fills the Draco-defined matrix, or perhaps has already
done so during the call to the matrix's constructor.

\begin{ttfamily}
\cxxcode
  NewMatrix *matrix = 
    MatrixFactoryTraits<NewMatrix>::create(myDracoMatrix);
\end{ttfamily}

At this point the issues of restricting matrix conversions come into
play.
We may have a linear system defined by the diffusion
solver for a totally unstructured mesh, expressed by Draco's compressed row
storage representation (CRS), and a matrix defined for a
nine-point structured linear system.
In this case, depending on the intent of the designer of the
\code{MatrixFactoryTraits} specialization,
we may not allow the creation of nine-point structured matrix
from Draco's CRS representation, or allow the creation
of the nine-point matrix, subject to the CRS matrix having the same
sparsity pattern as the nine-point matrix.
If the designer disallows the conversion, then the above line of code
will result in a \emph{compile time} failure.
If the designer allows the conversion, then the above line of code
will result in a possible \emph{run time} failure, along with
a thrown exception.

The following main program demonstrates how the common
\code{MatrixFactoryTraits} mechanism is used.
In this case the \code{DenseMatrixRep} is a Draco-defined matrix
representation, and the \code{JoubertMat} is a matrix defined by a
linear algebra package.

\subsection{main.cc}

\begin{ttfamily}
\begin{small}
\cxxcode
\input{main.cc.cfg}
\end{small}
\end{ttfamily}

\section{How the \code{MatrixFactoryTraits} mechanism is implemented.}

\subsection{MatrixFactoryTraits.hh}

\begin{ttfamily}
\begin{small}
\cxxcode
\input{MatrixFactoryTraits.hh.cfg}
\end{small}
\end{ttfamily}

\section{How the \code{MatrixFactoryTraits} mechanism is specialized.}

\subsection{JoubertMatTraits.hh}

\begin{ttfamily}
\begin{small}
\cxxcode
\input{JoubertMatTraits.hh.cfg}
\end{small}
\end{ttfamily}

\subsection{JoubertMatTraits.cc}

\begin{ttfamily}
\begin{small}
\cxxcode
\input{JoubertMatTraits.cc.cfg}
\end{small}
\end{ttfamily}

\bibliographystyle{apalike}

\bibliography{matFactory}

\closing
\end{document}

%%---------------------------------------------------------------------------%%
%% end of matFactory.tex
%%---------------------------------------------------------------------------%%
