%%---------------------------------------------------------------------------%%
%% draco_bs.tex
%% Thomas M. Evans
%% Time-stamp: <99/04/20 08:45:57 tme>
%%---------------------------------------------------------------------------%%
\documentclass[reqno]{larep}
\usepackage[dvips]{graphicx}
\usepackage{subfigure,epsfig}
\usepackage{cite}
\usepackage[centertags]{amsmath}
\usepackage{amssymb}
\usepackage[mathcal]{euscript}
\usepackage{latexsym}
\usepackage{tmath}
\usepackage{c++,fancycodes}
\usepackage{tabularx}
\usepackage{draco_bs}
\usepackage{alltt}
%\usepackage{doublespace}
%\usepackage[nomarkers]{endfloat}

\makeindex

%%---------------------------------------------------------------------------%%
%% BEGIN DOCUMENT
%%---------------------------------------------------------------------------%%

\begin{document}

%%---------------------------------------------------------------------------%%
%% Front Matter
%%---------------------------------------------------------------------------%%

\frontmatter

\title{The Draco Build System}
\author{T.M. Evans}
\author{R.M. Roberts}
\address{X--TM, MS D409, Los Alamos National Laboratory, Los Alamos, NM
  87544}
\email{tme@lanl.gov}
\email{rsqrd@lanl.gov}

\subjclass{LA-12345}
\date{\today}

\keywords{cvs, gmake, GNU, autoconf, gm4, compilers, KCC, ld}

\begin{abstract}
  
  We present a new build system for the Draco component library.  The
  build system is designed to facilitate both development and usage on 
  multiple platforms.  The build system complies with the GNU Coding
  Standards for software packages.  The build system has been designed 
  according to the following list of requirements:
  \begin{enumerate}
  \item support for simultaneous, multiple configurations;
  \item support for \cpp\ explicit template instantiation;
  \item support for multiple languages;
  \item support for \dejagnu\ and regression testing;
  \item support for \purify;
  \item compliance with the GNU coding standard.
  \end{enumerate}
  These requirements plus additional features such as parallel
  building and vendor support have been included in the \draco\ build
  system. 
  
  The build system uses four GNU tools, \autoconf, \gmake, \gmfour,
  and \dejagnu.  Version control of the \draco\ source is performed by
  \cvs.  These tools are all freely available from the Free Software
  Foundation.

\end{abstract}
\maketitle

\tableofcontents
\listoffigures
\listoftables

%%---------------------------------------------------------------------------%%
%% Main Matter
%%---------------------------------------------------------------------------%%

\mainmatter

%%---------------------------------------------------------------------------%%
%% introduction for imc-dev manual
%%---------------------------------------------------------------------------%%

\section{Introduction}

The \imctest package represents a first step towards the development
of \milagro.  In this sense, \imctest may be considered an early
release of \milagro.  The goals of \imctest are:
\begin{enumerate}
\item develop an extensible C++ object-oriented/generic design;
\item test the implementation of the IMC algorithm on several simple
  test problems;
\item test the parallel performance of IMC on several platforms.
\end{enumerate}
The object-oriented/generic design of \imctest will ensure an easy
transition to the full \milagro package.  The improvements and
additions that are required to promote \imctest to \milagro include:
\begin{enumerate}
\item interfaces to various host-codes;
\item improved physics;
\item new mesh types and geometries;
\end{enumerate}
\input{model}
\input{compile}
\input{extern}
\input{adding}
\input{extend}

%%---------------------------------------------------------------------------%%
%% Back Matter
%%---------------------------------------------------------------------------%%

\backmatter

\appendix

\input{vendors}

\bibliographystyle{../tex/rnote}
\bibliography{../bib/draco}

\printindex

\end{document}

%%---------------------------------------------------------------------------%%
%% end of draco_bs.tex
%%---------------------------------------------------------------------------%%


