%%---------------------------------------------------------------------------%%
%% overview for imc-dev manual
%%---------------------------------------------------------------------------%%

\section{Overview}

This section describes both the philosophy behind the \imctest\ code
design and the code organization.  The section concludes with a
summary of the code manual organization.

\subsection{The \draco\ Philosophy}

As mentioned previously, \imctest\ is built under \draco.  \draco\ is a
\cpp\ framework which provides a general structure for building
computational physics codes.  It's purpose is organizational as
opposed to being a hard and fast framework.  The principle \draco\ services
used by \imctest\ include:
\begin{itemize}
\item \dspp, a \cpp\ services package that includes \comp{SP} (smart
  pointer), \comp{Mat} (matrix), and \comp{Assert} classes;
\item \cccc, a \cpp\ communications interface package that allows
  concise message passing using MPI.
\end{itemize}
In addition, \draco\ maintains a multifunctional build model which
performs explicit instantiation of template classes efficiently.

The underlying principle of \draco\ is the development of transport
libraries which are parameterized on mesh-types (MT).  In \cpp\ this
means defining class templates which may be parameterized on (nearly)
any MT.  \imctest\ follows this organizational principle. 

\subsection{\imctest\ Library Class Organization}

The classes used in \imctest\ can be divided into three major modules:
\mle{Pre-Process}, \mle{Solver}, and \mle{Post-Process}.  In addition,
we have a service module, \mle{Global}.  In reality, \mle{Global} is a
namespace, \comp{IMC::Global}, that holds inline, non-member,
templated, service functions.  It also contains fundamental and
physical constants.  Within each module there are sub-modules which
organize the classes into logical structures. The classes contained in
each module and sub-module are displayed in Fig.~\ref{fig:classes}.
\begin{figure}
  \begin{center}
    \begin{tabular}{c}
      \vspace{1\baselineskip} \\
      \begin{tabular}{|l|l|} \hline
        \multicolumn{1}{|c|}{\mle{Interface}} & 
        \multicolumn{1}{c|}{\mle{Builder}} \\\hline
        \comp{OS\_Interface} & \comp{OS\_Builder} \\
        & \comp{Opacity\_Builder\tlate{MT}} \\
        & \comp{Source\_Builder\tlate{MT}} \\
        & \comp{Parallel\_Builder\tlate{MT}} \\\hline
      \end{tabular} \\
      (a) \\
      \vspace{1\baselineskip} \\
      \begin{tabular}{|l|l|l|}\hline
        \multicolumn{1}{|c|}{\mle{Mesh}} &
        \multicolumn{1}{c}{\mle{Material}} &
        \multicolumn{1}{|c|}{\mle{Transport}} \\\hline
        \comp{OS\_Mesh} &  \comp{Mat\_State\tlate{MT}} &
        \comp{Source\tlate{MT, RN}} \\
        \comp{Coord\_sys} & \comp{Opacity\tlate{MT}} &
        \comp{Particle\tlate{MT, RN}} \\
        \comp{Layout} & & \comp{Random} \\
        & & \comp{Particle\_Stack\tlate{PT}} \\
        & & \comp{Tally\tlate{MT}} \\\hline
      \end{tabular} \\
      (b) \\
      \vspace{1\baselineskip} \\
      \begin{tabular}{|l|} \hline
        \multicolumn{1}{|c|}{\mle{Output}} \\ \hline
        \comp{Assemble\tlate{MT}} \\
        \comp{Output\tlate{MT}} \\\hline
      \end{tabular} \\
      (c) \\
    \end{tabular}
  \end{center}
  \caption{Module composition in \imctest, (a) the \mle{Pre-Process} 
    module, (b) the \mle{Solver} module, and (c) the
    \mle{Post-Process} module.}
  \label{fig:classes}
\end{figure}

The \mle{Pre-Process} module controls the operations that are required 
at the beginning of each calculational cycle.  The \mle{Interface}
sub-module is responsible for interfacing to the host, which in
\imctest\ is an input file.  The \mle{Build} sub-module creates the
objects necessary for performing IMC transport across multiple
processors.

The \mle{Solver} module actually performs the IMC transport using
objects created by the \mle{Pre-Process::Build} module.  The
\mle{Mesh} sub-module contains the various mesh-types upon which
\imctest\ is designed.  The \mle{Material} sub-module defines the
material state and opacities.  Finally, the \mle{Transport} sub-module
performs the IMC calculation on the geometry specified by the
\mle{Mesh} and \mle{Material} sub-modules.

The output and end-of-cycle processing is controlled by the
\mle{Post-Process} module.  The \mle{Output} sub-module takes the
transport output from each processor and assembles the data for
output.  It also prepares the problem for the next cycle by storing
the census particles.

The files listed in Fig.~\ref{fig:classes} compose the \imctest\ 
library.  The \imctest\ library is built under \draco.  \imctest\ 
also provides an execution class, \comp{IMC\_run}, that utilizes the
modules in the \imctest\ library and creates an \imctest\ stand-alone
executable.

\subsection{Code Manual Organization}

The following sections will describe the classes listed in the
preceding section.  We begin by explaining the mesh-types available in 
\imctest.  We begin with the \mle{Solver::Mesh} module because
everything in \imctest\ (and \draco) is parameterized on a MT.  From
there, we continue by illucidating the rest of the \mle{Solver}
modules.  After these sections, we will describe the \mle{Pre-Process} 
and \mle{Post-Process} modules.
