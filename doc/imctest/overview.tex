%%---------------------------------------------------------------------------%%
%% overview for imc-dev manual
%%---------------------------------------------------------------------------%%

\section{Overview}

As mentioned previously, \imctest\ is built under \draco.  \draco\ is a
\cpp\ framework which provides a general structure for building
computational physics codes.  It's purpose is organizational as
opposed to being a hard and fast framework.  The principle \draco\ services
used by \imctest\ include:
\begin{itemize}
\item \dspp, a \cpp\ services package that includes \comp{SP} (smart
  pointer), \comp{Mat} (matrix), and \comp{Assert} classes;
\item \cccc, a \cpp\ communications interface package that allows
  concise message passing using MPI.
\end{itemize}
In addition, \draco\ maintains a multifunctional build model which
performs explicit instantiation of template classes efficiently.

The underlying principle of \draco\ is the development of transport
libraries which are parameterized on mesh-types (MT).  In \cpp\ this
means defining class templates which may be parameterized on (nearly)
any MT.  \imctest\ follows this organizational principle. 

The classes used in \imctest\ can be divided into four major modules:
\mle{Builder}, \mle{Interface}, \mle{Transport}, and \mle{Global}.  In
reality, \mle{Global} is not a class module; instead, it is a
namespace, \comp{IMC::Global}, that holds inline, non-member,
templated, service functions.  It also contains fundamental and
physical constants.  The classes contained in each module are
exhibited in Table~\ref{tab:classes}.
\begin{table}
  \begin{center}
    \caption{Module composition in \imctest.}
    \label{tab:classes}
    \begin{tabular}{llll} \hline\hline
      \multicolumn{4}{c}{Modules} \\
      \multicolumn{1}{c}{\mle{Interface}} & \multicolumn{1}{c}{\mle{Builder}} 
      & \multicolumn{1}{c}{\mle{Transport}} & \multicolumn{1}{c}{\mle{Global}}
      \\ \hline\hline
      \comp{OS\_Interface} & \comp{OS\_Builder} & \comp{OS\_Mesh} &
      Constants.hh \\
      & \comp{Opacity\_Builder\tlate{MT}} & \comp{Coord\_sys} & Math.hh \\
      & \comp{Source\_Init\tlate{MT}} & \comp{Layout} & \\
      & \comp{Parallel\_Builder\tlate{MT}} & \comp{Source\tlate{MT}} & \\
      & & \comp{Opacity\tlate{MT}} &\\
      & & \comp{Mat\_State\tlate{MT}} & \\
      & & \comp{Particle\tlate{MT}} & \\
      & & \comp{Random} & \\
      & & \comp{Tally\tlate{MT}} & \\ \hline\hline
    \end{tabular}
  \end{center}
\end{table}

The files listed in Table~\ref{tab:classes} compose the \imctest\ 
library.  The \imctest\ library is built under \draco\.  \imctest\ 
also provides an execution class, \comp{IMC\_run}, that utilizes the
modules in the \imctest\ library and creates an \imctest\ stand-alone
executable.

